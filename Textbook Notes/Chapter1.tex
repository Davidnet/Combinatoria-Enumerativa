\chapter{Basic Methods}
\subsection{When we add}
\begin{thm}
	\textbf{Addition Priniciple} If $A$ and $B$ are two disjoint finite sets, then:
	\[ \abs{ A \cup B} = \abs{A} + \abs{B} \]
\end{thm}
\begin{proof}
	Both sides of the above relation count the elements of the same set, the set $ A \cup B$. The left-hand side does this directly, while the right-hand side counts the elements of $A$ and $B$ separately. In either case, each element is counted exactly once (as $A$ and $B$ are disjoint), so the two sides are indeed equal
\end{proof}
Observe that the previous theorem was about two disjoint finite sets.
\begin{thm}
	\textbf{Generalized Addition Principle} Let $ A_1, A_2, \ldots A_n$ be finite sets that are pairwise disjoint. Then
	\begin{align*}
	\abs{A_1 \cup A_2 \cup \ldots A_n} = \abs{A_1} + \abs{A_2} + \ldots \abs{A_n}
	\end{align*}
\end{thm}
\begin{proof}
	Again, both sides count the elements of the same set, the set $ A_1 \cup A_2 \cup \ldots A_n $, therefore they have to be equal.
\end{proof}
\section{When We Subtract}
For this section we will use the following definition
\begin{define}
	\textbf{Difference of two sets} If $A$ and $B$ are two sets, then $A - B$ is the set consisting of the elements of $A$ that are not elements of $B$
\end{define}
Although the difference is defined for $B \not \subset A$, we will only consider cases on which $ B \subseteq A$.
\begin{thm}
	\textbf{Subtraction Priniciple} Let $A$ be a finite set, and let $ B \subseteq A$. Then $ \abs{ A - B} = \abs{A} - \abs{B}$
\end{thm}
\begin{proof}
	On a more easy way, let us prove the equivalent relation:
	\[ \abs{A - B} + \abs{B} = \abs{A} \]
	This relation holds true by the Addition Principle. Indeed, $A - B$ and $B$ are disjoint set that their union is $A$.
\end{proof}
An important remark to denote here, is the fact that the hypothesis of $ B \contained A $ is an important restriction.

The use of the Subtraction Principle is advisable in situations when it is easier to enumerate the elements of $B$ (``bad guys'') than the elements of $ A - B $ (`` good guys '' ). As a general rule of thumb, it is easy to compute $\abs{A}$ and $\abs{B}$ rather than compute $ \abs{A - B} $ directly.

\textbf{When We Multiply}

\textbf{The Product Principle}

\begin{thm}
	\textbf{Product Principle} Let $X$ and $Y$ be two finite sets. Then the number of pairs $(x,y)$ satisfying $x \in X $ and $y \in Y$ is $\abs{X} \times \abs{Y}$
\end{thm}
\begin{proof}
	There are $\abs{X} $ choices for the first element $x$ of the pair $(x,y)$, then regardless of what we choose for $x$, there are $\abs{Y}$ choices for $y$. Each choice of $x$ can be paired with each choice of $y$, so the statement is proved.
\end{proof}

A generalized version is the following:

\begin{thm}
	\textbf{Generalized Product Principle} Let $X_1,x_2, \ldots X_k$ be inite sets. Then, the number of k-tuples $(x_1,\ldots,x_k) $ satisfying $x_i \in X_i$ is $ \abs{X_1} \times \ldots \times \abs{X_k} $.
\end{thm}

\begin{proof}
	We prove the statement by induction on $k$. For $k = 1$, there is nothing to prove, and for $ k = 2$, the statement reduces to the Product Principle. Now let us assume that we know the statement for $ k-1$, and let us prove it for $k$. A k-tuple $ (x_1, \ldots, x_k) $ satisfying $x_i \in X_i$ can be decomposed into an ordered pair $ ((x_1, \ldots, x_{k-1}), x_k) $, where we still have $x_i \in X_i$. The number of such k-1-tuples ia by our induction hypothesis, $ \abs{X_1} \times \ldots \times \abs{X_{k-1}} $. The number of elements $x_k \in X_k$ is $ \abs{X_k} $. Therefore, by the Product Principle, the number of ordered pairs $ ((x_1, \ldots, x_{k-1}), x_k) $ satisfying the conditions is;
	\[ (\abs{X_1} \times \ldots \abs{X_{k-1}}) \times \abs{X_k} \]
	so this is also the number of k-tuples $ (x_1, \ldots x_k) $ satisfying $ x_i \in X_i $
\end{proof}

An special case of the Theorem above, is when all $ \abs{X_i} $ have the same size. If $A$ is a finite alphabet consisting of $n$ letters, then a k-letter string over $A$ is a sequence of k letters,each of which is an element of $A$.

\begin{cor}
	The number of k-letters strings over an n-element alphabet $A$ is $n^k$
\end{cor}
\begin{proof}
	Apply the above theorem.
\end{proof}

\textbf{Using Several Counting Principles}

\textbf{When Repetitions Are Not Allowed}

\textbf{Permutations}

\begin{define}
	Let $n$ be a positive integer. Then the number:
	\[ n \cdot (n-1) \ldots 2 \cdot 1 \]
	is called n-factorial, and is denoted by $n!$
\end{define}
We define $0! = 1$.
The set $ \set{1,2, \ldots, n} $ will be denoted with: $ [n] $ 
\begin{thm}
	For any positive integers $n$, the number of ways to arange all elements of the set $ [n] $ in a line is $n!$
\end{thm}
\begin{proof}
	There are $n$ ways to select the element that will be at the first place in our line. Then, regardless of this selection, there are $n-1$ ways to select the element that will be listed second, and so on \ldots.
\end{proof}
an important definition is on place:

\begin{define}
	A permutation of a finite set $S$ is a list of the elements of $S$ containing each element of $S$ exactly once.
\end{define}

\textbf{Partial Lists Without Repetition}

\begin{thm}
	Let $n$ and $k$ be positive integers so that $ n \geq k$. Then, the number of ways to make a k-element list from $[n]$ without repeating any elements is:
	\[ n(n-1)(n-2) \ldots (n-k+1) \]
\end{thm}
\begin{proof}
	There are $n$ choices for the first element of the list, then $n-1$ choices for the second element of the list, and so on; finally there are $n-k+1$ choices for the las (kth) element of the list. The result then follows by the Product Principle.
\end{proof}
As a notation abbreviation, we define as:
\[ (n)_k = n(n-1)(n-2)\ldots(n-k+1) \]



\textbf{When We Divide}

\textbf{The Division Principle}

\begin{define}
	Let $S$ and $T$ be finite sets, and let $d$ be a fixed positive integer. We say that the function $ f:T \rightarrow S $ is $d-$to-one if fir each element $s \in S$ there exist exactly $d$ elements $ t \in T$ so that $f(t) = s$
\end{define}

\begin{thm}
	\textbf{Division Principle} Let $S$ and $T$ be finite sets so that a $d-$to-one function $ f: T \rightarrow S $ exits. Then:
	\[ \abs{S} = \frac{\abs{T}}{d} \]
\end{thm}

\textbf{Example}

An important example: Let us ask $n$ people to sit around a circular table, and consider two seating arrangements identical if each person has the same \textit{left neighbor} in both seatings.

An important result is that: The number of different seating arrangements for $n$ people around a circular table is $ (n-1)! $

The solution is given as, If the table were linear, instead of circular, then the number of all seating arrangements would be $n!$. In other words, if $T$ is the set of seating arrangements of $n$ people along a linear table, then $ \abs{T} = n! $. Now let $s$ be the set of seating arrangements  around our circular table, we claim that each element of $S$ corresponds to $n$ elements of $T$. Indeed, take a circular seating $s \in S$, and choose a person $p$ in that seating, in one of $n$ ways. Then turn $s$ into a linear seating, by starting the seating with $p$, then continuing with the left neighbor of $p$, the left neighbor of that person, then the left neighbor of that person, and so on. This turns $s$ into a linear seating. As there are $n$ choices for $p$, each circular seating $s$ can be turned into $n$ different linear seating arrangements.
On the other hand, each linear seating $ \operatorname{lins} $ corresponds to one circular seating $f(\operatorname{lins})$, because no matter where we ``fold'' lins into a circle, the left neighbor of each person will not change.
This means that $ f:T \rightarrow S $ is an $n$-to-one function. Therefore, by the Division Principle, 
\[ \abs{S} = \frac{\abs{t}}{N} = \frac{n!}{n} = (n-1)! \]
So this is the number of circular seating arrangements.



 
