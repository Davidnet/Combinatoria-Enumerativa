\chapter{Basic Methods}
\subsection{When we add}
\begin{thm}
	\textbf{Addition Priniciple} If $A$ and $B$ are two disjoint finite sets, then:
	\[ \abs{ A \cup B} = \abs{A} + \abs{B} \]
\end{thm}
\begin{proof}
	Both sides of the above relation count the elements of the same set, the set $ A \cup B$. The left-hand side does this directly, while the right-hand side counts the elements of $A$ and $B$ separately. In either case, each element is counted exactly once (as $A$ and $B$ are disjoint), so the two sides are indeed equal
\end{proof}
Observe that the previous theorem was about two disjoint finite sets.
\begin{thm}
	\textbf{Generalized Addition Principle} Let $ A_1, A_2, \ldots A_n$ be finite sets that are pairwise disjoint. Then
	\begin{align*}
	\abs{A_1 \cup A_2 \cup \ldots A_n} = \abs{A_1} + \abs{A_2} + \ldots \abs{A_n}
	\end{align*}
\end{thm}
\begin{proof}
	Again, both sides count the elements of the same set, the set $ A_1 \cup A_2 \cup \ldots A_n $, therefore they have to be equal.
\end{proof}
\section{When We Subtract}
For this section we will use the following definition
\begin{define}
	\textbf{Difference of two sets} If $A$ and $B$ are two sets, then $A - B$ is the set consisting of the elements of $A$ that are not elements of $B$
\end{define}
Although the difference is defined for $B \not \subset A$, we will only consider cases on which $ B \subseteq A$.
\begin{thm}
	\textbf{Subtraction Priniciple} Let $A$ be a finite set, and let $ B \subseteq A$. Then $ \abs{ A - B} = \abs{A} - \abs{B}$
\end{thm}
\begin{proof}
	On a more easy way, let us prove the equivalent relation:
	\[ \abs{A - B} + \abs{B} = \abs{A} \]
	This relation holds true by the Addition Principle. Indeed, $A - B$ and $B$ are disjoint set that their union is $A$
\end{proof}

